\documentclass{article}
\usepackage[utf8]{inputenc}
\usepackage[spanish]{babel}
\usepackage{lipsum}
\title{Mi primer  documento en \LaTeXe}
\author{Víctor Hugo Vázquez Montoya}
\date{\small{\today}}

\begin{document}

\maketitle

\section{Introducción}
\subsection{Animales marinos}
\subsubsection{Características}
%\lipsum[2-4]
{\Huge \textbf{\underline{\LaTeX}}}
Se denominan \emph{animales marinos} a todos \textbf{\textit{aquellos animales} que} viven en las \textbf{aguas} del \underline{mar}, toda su vida o al menos gran parte de ella. A continuación, te explicamos a fondo todo de los

%\vspace{1cm}
\newpage

{\Huge \textbf{\underline{\LaTeX}}}
Se denominan \emph{animales marinos} a todos \textbf{\textit{aquellos animales} que} viven en las \textbf{aguas} del \underline{mar}, toda su vida o al menos gran parte de ella. A continuación, te explicamos a fondo todo de los\\
{\Large Tipos de animales}
 \begin{itemize}
     \item Mamíferos
     \item Vivíparos
     \item Ovíparos
 \end{itemize}
 {\footnotesize Ejemplos}
 \begin{enumerate}
     \item Vaca
     \item Gato
    \item Pollo
 \end{enumerate}
 \section{Entornos matemáticos}
 \begin{enumerate}
     \item Resuelve las siguientes derivadas por el método de los cuatro pasos
     \begin{itemize}
         \item [(a)]$f(x)=x^2+10x$
         \item [(b)] $y=\frac{e^{2x+10}+\sin x}{ln(\cos x)}$
         \item [(c)] $x_{1}=6,x_{2}=7,\ldots, x_{10}=9$
     \end{itemize}
 \end{enumerate}
\end{document}
